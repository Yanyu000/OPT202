\documentclass[12pt]{article}
\usepackage{amsmath}
\usepackage{graphicx}
\usepackage[margin=1in,footskip=0.25in]{geometry}
\usepackage{hyperref}
\usepackage[utf8]{inputenc}
\title{OP202 Homework}
\author{Yanyu ZHOU}
\date{\today}
\begin{document}
\maketitle
\section{}
\textbf{Lemma 3: }Function $f$ belongs to $\mathcal{C}^{2,1}_L(\mathbf{R}^n)$ if and only if
$$|| \nabla^2 f(x) ||\leq L, \quad \forall x\in\mathbf{R}^n.$$
\textbf{Proof:} 

For one direction, if $f \in \mathcal{C}^{2,1}_L(\mathbf{R}^n)$, we have
\begin{itemize}\label{C21}
    \item $f$ is 2 times continuously differentiable in $\mathbf{R}^n$.
    \item its first derivative is Lipschitz continuous in $\mathbf{R}^n$ with the constant $L$: $$||\nabla f(y)-\nabla f(x)||\leq L||y-x|| .$$
\end{itemize}
Thus, by the definition of derivative and Lipschitz condition, for any $s \in \mathbf{R}^n, \alpha \in \mathbf{R}^+$ we have
$$||(\int\limits_0^\alpha \nabla^2f(x+\tau s)d\tau )s|| = ||\nabla f(x+\alpha s)-\nabla f(x)|| \leq \alpha L||s||.$$
Taking the limit of $\alpha$ to 0, we obtain: 
$$\lim_{\alpha \rightarrow 0} \frac{||\nabla f(x+\alpha s)-\nabla f(x)||}{\alpha} \leq  L||s||,$$
$${\boxed{  || \nabla^2 f(x) ||\leq L}}.$$


For the other direction, by Mean value theorem, $\forall x,y \in \mathbf{R}^n$
$$\nabla f(y) =  \nabla f(x) + \int\limits_0^1 <\nabla^2f(x+ \tau (y-x)), y-x > d \tau $$
Thus, with the condition that the second derivative of $f$ is bounded, 
\begin{equation}
\begin{split}
||\nabla f(y)-\nabla f(x)|| &= ||\int\limits_0^1 <\nabla^2f(x+ \tau (y-x)), y-x > d \tau ||\\
    & \leq  \int\limits_0^1 ||<\nabla^2f(x+ \tau (y-x)), y-x > ||d \tau\\
    & \underset{\text{C.S.}}{\leq} \int\limits_0^1 ||\nabla^2f(x+ \tau (y-x)) ||\  ||y-x|| d \tau\\
    & \leq \int\limits_0^1 ||\nabla^2f(x+ \tau (y-x)) ||  d \tau||y-x|| \leq L||y-x|| 
\end{split}
\end{equation}

\textbf{Corollary: }Function $f$ belongs to $\mathcal{C}^{p+1,p}_L(\mathbf{R}^n)$ if and only if
$$|| \nabla^{p+1} f(x) ||\leq L, \quad \forall x\in\mathbf{R}^n.$$

\textbf{Proof:} 
For one direction, if $f \in \mathcal{C}^{p+1,p}_L(\mathbf{R}^n)$, for any $s \in \mathbf{R}^n, \alpha >0$ we have
$$||(\int\limits_0^\alpha \nabla^{p+1}f(x+\tau s)d\tau )s|| = ||\nabla^p f(x+\alpha s)-\nabla^p f(x)|| \leq \alpha L||s||.$$
Taking the limit of $\alpha$ to 0, we obtain: 
$$\lim_{\alpha \rightarrow 0} \frac{||\nabla^p f(x+\alpha s)-\nabla^p f(x)||}{\alpha} \leq  L||s||,$$
$${\boxed{  || \nabla^{p+1} f(x) ||\leq L}}.$$

For the other direction, $\forall x,y \in \mathbf{R}^n$
$$\nabla^p f(y) =  \nabla^p f(x) + \int\limits_0^1 <\nabla^{p+1}f(x+ \tau (y-x)), y-x > d \tau $$
Thus, 
\begin{equation}
\begin{split}
||\nabla^p f(y)-\nabla^p f(x)|| &= ||\int\limits_0^1 <\nabla^{p+1}f(x+ \tau (y-x)), y-x > d \tau ||\\
    & \leq  \int\limits_0^1 ||<\nabla^{p+1}f(x+ \tau (y-x)), y-x > ||d \tau\\
    & \underset{\text{C.S.}}{\leq} \int\limits_0^1 ||\nabla^{p+1}f(x+ \tau (y-x)) ||\  ||y-x|| d \tau\\
    & \leq \int\limits_0^1 ||\nabla^{p+1}f(x+ \tau (y-x)) ||  d \tau||y-x|| \leq L||y-x|| 
\end{split}
\end{equation}

\textbf{Remark: } 
\begin{itemize}
    \item The Lipschitz condition actually comes many time from the Mean value theorem:
    $$||f(y)- f(x)||\leq ||\nabla f((1-\tau)x+\tau y)||\ ||y-x||.$$
    \item $\nabla f$ is a vector, $\nabla^2 f$ is a Hessian matrix, $\nabla^p f $ is a tensor when $p \geq 3$.
\end{itemize}
\end{document}