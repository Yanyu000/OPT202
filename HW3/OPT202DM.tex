\documentclass[12pt]{article}
\usepackage{lipsum}
\usepackage{amsmath}
\usepackage{amsthm}

\newtheoremstyle{mystyle}%                % Name
  {}%                                     % Space above
  {}%                                     % Space below
  {\itshape}%                             % Body font
  {}%                                     % Indent amount
  {\bfseries}%                            % Theorem head font
  {.}%                                    % Punctuation after theorem head
  { }%                                    % Space after theorem head, ' ', or \newline
  {}%                                     % Theorem head spec (can be left empty, meaning `normal')

\theoremstyle{mystyle}
\newtheorem{thm}[equation]{Theorem}
\newtheorem{cor}[equation]{Corollary}
\newtheorem{lemma}[equation]{Lemma}
\newtheorem{prop}[equation]{Proposition}
\newtheorem{conj}[equation]{Conjecture}
\newtheorem{definition}[equation]{Definition}
\newtheorem{remark}[equation]{Remark}
\newtheorem{example}[equation]{Example}

\usepackage{graphicx}
\usepackage[margin=1in,footskip=0.25in]{geometry}
\usepackage{hyperref}
\usepackage[utf8]{inputenc}
\title{OP202 HW3}
\author{Yanyu ZHOU}
\date{\today}

\begin{document}
\maketitle
\section{Q1}
Consider
\begin{equation} \label{PC}
\tag{$PC$} 
    \min _{x \in X} f(x)
\end{equation}
\begin{center}
    $\Updownarrow$
\end{center}
\begin{equation}
    \tag{$\mathrm{PS}^{\prime}$}
    \min _{x \in \mathbf{R}^n} f(x)+\iota_X(x)
\end{equation}
where $\iota_X(x)$ is a indicator function. When $x\in X$, $\iota_X(x) = 1$, else $\iota_X(x) = \infty$.

\textbf{Proof:}$\partial \iota_X(x) \equiv N_X(x)$

For \eqref{PC}, the necessary and sufficient conditions for optimality is
$$
\partial f\left(x^*\right)+N_X\left(x^*\right) \ni \mathbf{0}
$$
where $N_X(x)$ is the normal cone operator of $X$ at $x$. By definition, the \textit{normal cone} of a set $\mathbf{X}$ at a boundary point $x_0$ is the set of all vectors
$y$ such that $y^T(x-x_0) \geq 0$ for all $x \in \mathbf{X}$.

By definition, $\partial \iota_X(x) = \{g \in \mathbf{R}^n: g^Tx \geq g^Ty \quad \forall y\in \mathbf{X} \}$.
Therefore, $\partial \iota_X(x) \equiv N_X(x)$.

\section{Q2}
Work out one of the proofs of Theorem 17 (the simplest one is $f_1 \in \mathbf{S}^{1,1}_{m,L}$).
Here we prove that Theorem 12 in class 2 is also valid for the proximal gradient method. Referrence: CMU 10-725 Lecture 8.

For $f(x) = f_1(x) + f_2(x)$, we assume that 
\begin{itemize}
    \item $f_1$ is convex and at least $f_1 \in \mathbf{S}^{1,1}_{0,L}$
    \item $f_2$ is convex, $\operatorname{prox}_{\alpha h} (x)=\arg \min\limits_z\left\{\frac{\|x-z\|^2}{2 \alpha}+h(z)\right\}$ can be evaluated.
\end{itemize}

\textbf{Notation: }The update step of proximal gradient method can be written as
$$x_k = x_{k-1} - t_kG_{t_k}(x_{k-1})$$ 
where $G_t(x)$ is the generalized gradient and is given by 
$$G_t(x)= \frac{x- \operatorname{prox}_t (x - t\nabla f_1(x))}{t}.$$

Then the following holds:\\
\textbf{Theorem:} The proximal gradient method with constant step size $\alpha \leq \frac{2}{L}$
has an ergodic(global) convergence certificate of 
$$
f\left(\bar{x}_t\right)-f\left(x^*\right) \leq \frac{C}{t+1}\left\|x_0-x^*\right\|^2
$$
for $\bar{x}_t=\frac{1}{t+1} \sum_{k=0}^t x_k, \text { and } C>0$.

\textbf{Proof:}
We begin by showing that 
$$f(y) \leq f_1(x) + \nabla g(x)^T(y-x) + \frac{L}{2}\left\|y-x\right\|^2 + f_2(y) \quad \forall x,y.$$
Since $\nabla f_1$ is Lipschitz with constant L, $\nabla^2 f_1 $

\textbf{Remark:} 
\begin{itemize}
    \item This has the same convergence rate $O(\frac{1}{t})$ as that of gradient descent.
\end{itemize}

\section{Q2}
Depending on $f$, you can derive conditions for which the dual ascent is converging depending on $\alpha$ and $f$ you can have linear convergence, or $O(1 / t), O\left(1 / t^2\right), O(1 / \sqrt{t}) \ldots$

\section{Q3}
Homework. Can you think of a dual Newton’s method? Can you think of a
dual proximal method? When would you apply these methods and with
which guarantees?

\end{document}