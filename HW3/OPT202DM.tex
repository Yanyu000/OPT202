\documentclass[12pt]{article}
\usepackage{lipsum}
\usepackage{amsmath}
\usepackage{amsthm}

\newtheoremstyle{mystyle}%                % Name
  {}%                                     % Space above
  {}%                                     % Space below
  {\itshape}%                             % Body font
  {}%                                     % Indent amount
  {\bfseries}%                            % Theorem head font
  {.}%                                    % Punctuation after theorem head
  { }%                                    % Space after theorem head, ' ', or \newline
  {}%                                     % Theorem head spec (can be left empty, meaning `normal')

\theoremstyle{mystyle}
\newtheorem{thm}[equation]{Theorem}
\newtheorem{cor}[equation]{Corollary}
\newtheorem{lemma}[equation]{Lemma}
\newtheorem{prop}[equation]{Proposition}
\newtheorem{conj}[equation]{Conjecture}
\newtheorem{definition}[equation]{Definition}
\newtheorem{remark}[equation]{Remark}
\newtheorem{example}[equation]{Example}

\usepackage{graphicx}
\usepackage[margin=1in,footskip=0.25in]{geometry}
\usepackage{hyperref}
\usepackage[utf8]{inputenc}
\title{OP202 HW3}
\author{Yanyu ZHOU}
\date{\today}

\begin{document}
\maketitle
\section{Q1}
Consider
\begin{equation} \label{PC}
\tag{$PC$} 
    \min _{x \in X} f(x)
\end{equation}

\begin{center}
    $\Updownarrow$
\end{center}
\begin{equation}
    \tag{$\mathrm{PS}^{\prime}$}
    \min _{x \in \mathbf{R}^n} f(x)+\iota_X(x)
\end{equation}
where $\iota_X(x)$ is an indicator function. When $x\in X$, $\iota_X(x) = 1$, else $\iota_X(x) = \infty$.

\begin{thm}
    $\partial \iota_X(x) \equiv N_X(x)$
\end{thm}

\begin{proof}
For \eqref{PC}, the necessary and sufficient conditions for optimality is
$$
\partial f\left(x^*\right)+N_X\left(x^*\right) \ni \mathbf{0}
$$
where $N_X(x)$ is the normal cone operator of $X$ at $x$. By definition, the \textit{normal cone} of a set $\mathbf{X}$ at a boundary point $x_0$ is the set of all vectors
$y$ such that $y^T(x-x_0) \geq 0$ for all $x \in \mathbf{X}$.

By definition, $\partial \iota_X(x) = \{g \in \mathbf{R}^n: g^Tx \geq g^Ty \quad \forall y\in \mathbf{X} \}$.
Therefore, $\partial \iota_X(x) \equiv N_X(x)$.
\end{proof}

\section{Q2}
Work out one of the proofs of Theorem 17 (the simplest one is $f_1 \in \mathbf{S}^{1,1}_{m,L}$).
Here we prove that Theorem 12 in class 2 is also valid for the proximal gradient method. Referrence: CMU 10-725 Lecture 8.

For $f(x) = f_1(x) + f_2(x)$, we assume that 
\begin{itemize}
    \item $f_1$ is convex and at least $f_1 \in \mathbf{S}^{1,1}_{0,L}$
    \item $f_2$ is convex, $\operatorname{prox}_{\alpha h} (x)=\arg \min\limits_z\left\{\frac{\|x-z\|^2}{2 \alpha}+f_2(z)\right\}$ can be evaluated.
\end{itemize}

\textbf{Notation: }The update step of proximal gradient method can be written as
$$x_k = x_{k-1} - \alpha G_{\alpha_k}(x_{k-1})$$ 
where $G_\alpha (x)$ is the generalized gradient and is given by 
$$G_\alpha (x)= \frac{x- \operatorname{prox}_\alpha (x - \alpha \nabla f_1(x))}{\alpha}.$$

\begin{thm}The proximal gradient method with constant step size $\alpha \leq \frac{2}{L}$ satisfies
    $$
    f\left(x_k\right)-f\left(x^*\right) \leq \frac{\left\|x_0-x*\right\|^2}{2 \alpha k}
    $$
where $x^*$ is the optimal solution.
\end{thm}
\begin{proof}
We begin by showing that 
$$f(y) \leq f_1(x) + \nabla g(x)^T(y-x) + \frac{L}{2}\left\|y-x\right\|^2 + f_2(y) \quad \forall x,y.$$
Since $f_1 \in \mathbf{S}^{1,1}_{0,L}$, by Taylor's expansion
\begin{align*}
    f_1(y) &= f_1(x)+ f_1(x)^T(y-x) + \frac{1}{2}(x-y)^T\nabla^2f_1(x)(x-y)\\
    &\leq f_1(x) + f_1(x)^T(y-x)  + \frac{L}{2} \left\|y-x\right\|^2
\end{align*}
Therefore
$$f(y) = f_1(y) + f_2(y) \leq f_1(x) + f_1(x)^T(y-x)  + \frac{L}{2} \left\|y-x\right\|^2 + f_2(y) .$$
Substituting $y = x_k = x_{k-1} - \alpha G_\alpha (x_{k-1}), x = x_{k-1}$ we have 
\begin{equation}\label{f1con}
    f(x_k) \leq f_1(x_{k-1}) - \alpha f_1(x_{k-1})^TG_\alpha (x_{k-1})  + \frac{L\alpha^2}{2} \left\|G_\alpha \right\|^2 + f_2(x_{k-1} - tG_\alpha(x_{k-1})) 
\end{equation}

Now 
\begin{align*}
    x_{k-1} - \alpha G_\alpha(x_{k-1}) &= \arg \min\limits_z\left\{\frac{\|x_k-z\|^2}{2 \alpha}+f_2(z)\right\}\\
    &= \arg \min\limits_z\left\{\frac{\|x_{k-1} - \alpha \nabla f_1(x_{k-1}) -z\|^2}{2 \alpha}+f_2(z)\right\}\\
    &= \arg \min\limits_z \left\{\frac{\|x_{k-1}^2 -z\|^2}{2 \alpha}+ \nabla f_1^T(z- x_{k-1})+ f_2(z)\right\}
\end{align*}
This implies that there exist a $\nu \in \partial f_2(z)$, $f_2(y) \geq f_2(z) + \nu^T(y-z),\ \forall y$, such that at minima $\nabla f_1(x) + \frac{1}{\alpha} (z-x) + \nu =0$,
but the minimum occurs at $z = x_{k-1} - \alpha G_\alpha (x_{k-1})$, thus
\begin{align*}
\nabla f_1(x_{k-1}) - G_\alpha (x_{k-1}) + \nu = 0, \nu \in \partial f_2(x_{k-1} - \alpha G_\alpha (x_{k-1}))\\
G_\alpha(x_{k-1}) - \nabla f_1(x_{k-1}) \in \partial f_2(x_{k-1} - \alpha G_\alpha (x_{k-1}))
\end{align*}

Since $f_2$ is convex, 
\begin{equation}\label{f2con}
    \begin{aligned}
    & f_2(x) \geq f_2\left(x_{k-1} - \alpha G_\alpha (x)\right)+\left(G_\alpha(x_{k-1})-\nabla f_1(x_{k-1})\right)^T \alpha G_t(x_{k-1}) \\
    & f_2\left(x_{k-1}- \alpha G_\alpha (x_{k-1})\right) \leq f_2(x_{k-1})-\alpha \left(G_\alpha (x_{k-1})-\nabla g(x_{k-1})\right)^T G_\alpha(x_{k-1})
    \end{aligned}
    \end{equation}

Substituting \eqref{f2con} in \eqref{f1con} we get
\begin{equation*}
    f(x_k) \leq f(x_{k-1}) - (1 - \frac{L \alpha^2}{2})\alpha \left\|G_\alpha(x_{k-1}) \right\|^2 
\end{equation*}

Since $f$ is convex as well, $f(x_{k-1}) \leq f\left(x^*\right)+G_\alpha (x_{k-1})^T\left(x_{k-1}-x^*\right)$, substituting it above we geometry
\begin{align*}
    f(x_k) &\leq f\left(x^*\right)+G_\alpha (x_{k-1})^T\left(x_{k-1}-x^*\right) - (1 - \frac{L \alpha^2}{2})\alpha \left\|G_\alpha(x_{k-1}) \right\|^2 \\
    &\leq f\left(x^*\right)+G_\alpha (x_{k-1})^T\left(x_{k-1}-x^*\right) -  \frac{ \alpha^2}{2} \left\|G_\alpha(x_{k-1}) \right\|^2\\
    &= f\left(x^*\right)+ \frac{1}{2\alpha} (\left\|x_{k-1} - x^* \right\|^2 - \left\|x_k - x^* \right\|^2 )
\end{align*}
\end{proof}

Summing over iteration we have:
\begin{equation}
    \begin{aligned}
    \sum_{i=1}^k\left(f\left(x_i-f\left(x^*\right)\right)\right. & \leq \frac{1}{2 \alpha}\left(\left\|x_0-x^*\right\|^2-\left\|x_k-x^*\right\|^2\right) \\
    & \leq \frac{1}{2 \alpha}\left\|x_0-x^*\right\|^2.
    \end{aligned}
    \end{equation}

Since the difference is non-increasing we have
\begin{equation}
f\left(x_k\right)-f\left(x^*\right) \leq \frac{1}{k} \sum_{i=1}^k f(x_i-f\left(x^*\right) ) \leq \frac{1}{2 \alpha k}\left\|x_0-x^*\right\|^2.
\end{equation}

\textbf{Remark:} 
\begin{itemize}
    \item This has the same convergence rate $O(\frac{1}{t})$ as that of gradient descent.
\end{itemize}

\section{Q3}
Depending on $f$, you can derive conditions for which the dual ascent is converging depending on $\alpha$ and $f$ you can have linear convergence, or $O(1 / t), O\left(1 / t^2\right), O(1 / \sqrt{t}) \ldots$
\begin{itemize}
    \item If $f$ is m-strongly convex, then the dual ascent with constant step size $\alpha = m$ converges with the rate of $O(1/t)$.
    \item If $f \in \mathbf{S}^{1,1}_{m,L}$, then the dual ascent with step size $\alpha = \frac{2}{1/m + 1/L}$ converges with the rate of $O(\log (1/t))$.
\end{itemize}
\section{Q4}
Homework. Can you think of a dual Newton’s method? Can you think of a
dual proximal method? When would you apply these methods and with
which guarantees?



\end{document}